
\chapter{Conclusions and Future Work}

\section{Conclusions}
In this document a pragmatic approach to deliver semantic technologies in present day web applications has been shown.
It has a simple formal model at its core that can be used to design and reasoning about web architectures. This model is
built blending the Pi Calculus for mobile process with the basic theory of tuple computing. Process calculi are
specially well suited for describing models where processes exchange channels of communication to other processes. The web can
be described as one of those models, where clients and servers exchange links to further servers. We have extended
these calculus with concepts from triple space computing, a version of the tuple space computing theory where triples
are accessed instead of arbitrary tuples. Since our main goal is the description of semantic web architectures, triples
as the one contained into RDF graphs, serve as the basic units for exchanging semantic data between the processes of our
calculus. Finally we have mapped the basic operations for accessing triples in triple space computing to the HTTP
interface and encoded these operations into the messages the processes of our version of the Pi Calculus can send and
receive. The result is a small model that can be used to describe web architectures in a formal way as a set of
semantic RESTful web services.\\

This theoretical model has been later derived into a specification where standard web technologies like RDF, RDF Schema,
SPARQL are used to propose a possible implementation of the model. The hRESTS-MicroWSMO specification proposal has been
used as the base of this specification. Extensions to this incipient standard has been proposed in this work dealing
with a possible binding of the services described to mechanisms for encoding the exchanged semantic data into a wire
syntax and for adapting the specification to particular environments, like the use of web browsers with Javascript engines
as the clients consuming RESTful web services.\\

This implementation is then delivered as a couple of software libraries for the server and client side. The server library allows developers to transform web
applications built with the Ruby on Rails framework into providers of RESTful semantic web services. The client library
can be used to provide a persistence layer for rich internet applications built in the Javascript language that are
backed by RESTful semantic web services. The choice of frameworks and languages has been driven by the desire of showing
how the proposed semantic technologies can be inserted into already existent applications built with Ruby on Rails or
any MVC framework in the client side, adding a semantic layer that can help building rich internet applications where
data can be easily exchanged between clients and servers.\\

These libraries have also been used to build Semantic Bespin, a sample web application that constitutes a true semantic mashup:
an application where a client with functionality similar to that of desktop applications consume semantic data from
different RESTful web services to aggregate them and deliver new functionality for its users. This application is a
sample of what can be achieved with services offering open semantic data that can be easily retrieved and manipulated
by web clients.\\

The provided libraries do not pretend to be the definitive solution for the problem of stagnation the Semantic Web is
suffering but can be used right now as a practical solution for building applications ranging from a
simple Javascript application that needs to
retrieve data from a server in a easy way, to a complex semantic application where it is required to feed an OWL reasoner.  Its
implementation is offered as open source and can evolve towards better solutions or just serve as the starting point for
new proposals of semantic technologies software libraries and tools.
The formal specification proposed in this work, and thus the software libraries implemented along this specification, are also deeply associated to
the hRESTS/microWSMO standard proposal. This is a very young proposal that still needs to mature. We have tried
to improve the specification in those aspects where no progress has been made yet, for instance, the actual binding of
the service model to concrete technologies for lifting and lowering semantic data.\\

The formal model proposed as the foundation of the work on the other hand, is not related in any aspect to a particular standard
proposal and can be used as the basis for other conceptions of service oriented architectures. The provided description
of this model has not completely being formalized but the conjunction of Pi Calculus and triple space computing can be
used in an easy way to describe formally many different distributed systems where semantic data is exchanged
between its components.


\section{Future work}


The work contained in this document can be considered in many aspects a preliminary work that can be further
expanded. The proposed formal model for describing RESTful semantic web services with concepts extracted from Pi
Calculus and triple space computing lacks complete formal specification.  These document provides only informal
semantics for the calculus and does not explore traditional applications of process calculi like bi-simulation.\\
Further investigation can also be pursued in the set of triple space operations that are used in the
calculus. Blocking operations and notifications modeled as notifications from the server to the clients were not
completely taken
into account since they do not fit into the REST architectural proposal. These operations nevertheless, are interesting
for describing less restrictive distributed computational environments where distinctions between clients and servers are
not that strict. In these architectures both roles are combined into the role of a agent or broker acting as client and
server. This kind of systems can be important in what is known as the �diffuse web�\cite{conf/coordination/Serrano09}, systems built with web
technologies where a user can, for instance, tune the temperature of her home from her mobile phone. Extensions to the
proposed theoretical model can be used to describe such systems where semantic data are can be used as the basic information units.\\

The realization of the proposed theoretical model using the hRESTS-MicroWSMO standard proposal will also need to be modified due to
the initial stage of development of this standard. Different standards for semantic web services have been proposed and
further standardization of web technologies can provide more convenient ways of binding semantic data to RESTful web
services.\\

The proposed solution for providing client and server support for RESTful semantic web services tried to be useful for
developers using present day frameworks like Ruby on Rails. New web frameworks trying to replicate the proposed
theoretical models in a more direct way can be explored. Functional concurrent oriented programming languages like
Erlang, and more recently Scala, Clojure and concurrent versions of Haskell can be used to build web frameworks where
requests are treated as a flow of exchanged semantic data between services, using triple repositories as the persistence layers. This
kind of systems have the potential to provide scalable architectures for future web development\\

Semantic REST and Siesta frameworks developed as part of this work can also be ported to mobile platforms like iPhone or Android, where application development
requires continuous exchange of data between users and remote web servers. RESTful semantic web services can provide an
open and reusable way of providing persistence to data used in mobile applications.\\

Another possible field of research is the use of semantic technologies like RDF triple graphs as the data model for
modern web applications where the persistence layer has moved away from relational data base management systems to
alternative persistence mechanisms like distributed key value stores. These persistence mechanisms share with
description logics and semantic data models like RDF the Open World assumption in contrast with the Closed World semantics of relational systems. The Open
World semantics allows greater flexibility in the description of data and semantic technologies and concepts could be
used in order to build a persistence layer on top of distributed key value stores.