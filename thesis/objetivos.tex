\chapter{Introduction}

World Wide Web Consortium (W3C) director's Tim Berners Lee, introduced the term Semantic Web in a seminal article in 2001\cite{berners-lee_semantic_2001} to designate his long term vision of what
the web should become. This vision involved the creation of a meta data layer on top of the existent web data describing the
meaning, the semantics, of these data. These semantic meta data would be necessary to enable automatic reasoning by software
web agents, that will retrieve and combine these data, inferring new facts when required, in order to fulfill the needs of the users
they work for. In the words of Berners Lee "The Semantic Web will bring structure to the meaningful content of Web
pages, creating an environment where software agents roaming from page to page can readily carry out sophisticated tasks
for users".\\

In the eight years that have passed since its original conception, a big effort to materialize this disruptive vision of
the web has been carried out. Organizations like the W3C, academic institutions and to a lesser degree, commercial
companies, have built foundational technologies that could enable the kind of applications envisioned by Berners Lee. 
Despite all these efforts, the Semantic Web remains today an ambitious idea yet to become true. The adoption of semantic
technologies by main stream web sites is small, semantic web applications are scarce and not well known, and even the
tools required for building semantic applications are not in a mature stage of development. This lack of success
combined with the initial high expectations set about what the Semantic Web could achieve has led to many people to consider
the Semantic Web just another buzzword that will never have an actual realization.\\

This state of things has originated the opinion that if the Semantic Web is to become a reality, it would be required a more
pragmatic solution to the problem of bridging the gap between today's web and the future Semantic Web. Before disruptive
semantic applications involving automatic reasoning can be a
reality, it is necessary that web applications start supporting semantic technologies situated at the base of the
semantic web technologies stack.  This will only be possible if the
entry barrier for the adoption of semantic technology is low and its benefits for the users or developers of a web
application real and immediate. Semantic technologies like annotated RDF (RDFa) that describes an easy mechanism for
inserting semantic meta data into HTML documents, that have met almost instant support by search engines, are steps into this direction. \\

This work tries to be a small contribution to this effort, describing a way for web developers and architects to design
and build their applications in terms of semantic web services that follows the core design principles of the HTTP
protocol. This will allow them to build, in a highly productive way, open systems with standard semantic meta data as
their main building blocks.

\section{The Semantic Web initiative and its promises}

The semantic web initiative was from its very conception a description of the web as a set of data resources that could be processed
in an automatic way by software applications. Under this conception, the main mechanism used by
software agents when processing data in the semantic web would be logical inference. The goal of enabling automatic
reasoning by software agents in the web has defined the main steps and technologies being developed under the semantic web
initiative. The outcome of this initiative has specified a set of layers on top of the HTTP protocol and the HTML mark up language, the core web
technologies, that make up a system for symbolic reasoning in the web. These layers include a formal syntax for facts provided
by XML, RDF and RDFS, an ontology description layer provided by OWL with roots in the description logics field and
current work in rules languages that can be used in conjunction with OWL like SWRL.\\

Using all these technologies it could be possible to develop web agents capables of achieving tasks that could only be
acomplished through the inference of new facts from data previously retrieved from the web. Classical examples of those tasks
are automatic biding and purchasing of goods, arrangement of appointments or the recommendation and filtering
of contents based on its semantic description.\\

Under this concdeption of the Semantic Web, additional layers are required for the building of semantic web applications and
agents. These layers involve dealing with the problem of uncertainty \cite{uncertaintyweb}. The proof and trust layers in the semantic web stack should provide mechanisms to enable
safe inference of new facts to web agents. Current research in this field include adopting theorical developments in
probability theory, fuzzy logic and belief functions.\\

At the top of semantic web initiative road map is the user interface and applications layer. This layer should provide
the mechanisms for a user to make actual use of the semantic web capabilities provided by the underlying semantic
technologies and standards.

\section{Current issues preventing the adoption of semantic technologies}

The Semantic Web initiative, as many attempts of drive radical technological change, is suffering from a bootstrap
problem. Current semantic technologies require widespread adoption to provide real benefits for their users, but users do not adopt
those technologies because they cannot be easily applied to present day web problems. \\

As we have mentioned, Semantic Web's development ultimate goal has been the achievement of automatic reasoning by software
agents in the web.  This vision of the web is in many ways radically different from today's web and so are the challenges
semantic web technologies like OWL and SPARQL try to solve. The problem most of these technologies face is that
while semantic web technological stack remains unfinished, adoption of these technologies has to come from nowadays
application developers. An important barrier preventing the adoption of semantic technologies by today's web developers is the relatively
complexity of these technologies. Technologies like OWL or rule systems require for their use the understanding of a
complex theoretical background that contrasts sharply with the simplicity of  main stream web technologies. \\

Another serious problem discouraging semantic web technologies adoption is the focus semantic technologies on symbolic
reasoning. Semantic standards provide the means to store facts in web documents that can be used in automatic inference,
but today's web is already fulfilled with unstructured data that cannot be easily annotated with semantic meta data. If these data are to
be used as a subject of knowledge for the Semantic Web they need to be processed in an automatic way, using statistical
techniques to produce annotated meta data with a certain degree of certainty. Another important concern is how to trust
facts exposed by semantic web data repositories. It is required that Semantic Web
technologies can cope with the concepts of uncertainty and probability, enabling the data mining of semantic data. \\

There is also a misunderstanding of use cases for semantic technologies. In absence of semantic web applications with
big user bases, there are no examples of how semantic technologies can be used to deliver more useful solutions. It is
necessary to find ways of applying semantic technologies to current issues in web development so they can serve as
example of semantic technology applicability capable of boosting further adoption.


\section{Lowercase Semantic Web}

As a reaction to the current situation of the semantic web, there has been a recent raise of interest in the use of
semantic web technologies to achieve improvements in problems found in current web application's design and
implementation. This kind of developments have been sometimes referred to as "lowercase semantic web''. This field was pioneered
by projects like the microformats project \url{http://microformats.org}, that tries to provide an easy way for
inserting semantics into HTML documents. They distinguished themselves by a stronger focus in delivering actual
improvements for web users, applying semantic technologies in pragmatic ways in today's web applications.\\

Search engines for instance, have started to parse semantic formats in web pages, as well as microformats, trying to obtain more accurate search
results for their users. Application developers are using microformats as a lightweight and easy to understand
alternatives to ontology languages like OWL, focusing in data exchange rather than automatic inference. \\

This pragmatic approach to the Semantic Web constitutes a bottom-up attempt of materializing the Semantic Web, building on
top of present day technologies, and trying to solve current problems, like the open exchange of data, with semantic standards.

\section{Opportunities for the semantic web}

Despite all these impediments, the current state of the web offers many chances to boost the adoption of semantic web
technologies. Current tendencies in the way of building web applications suppose an increase in the importance of data in the web.  Web
design is evolving at a fast pace from n-tier architectures where data persistence and business logic  were executed in the server and then rendered as HTML documents in the web browser, to a
model where business logic and rich user interfaces are executed in the browser and only persistence of data is provided
by the server. We are moving from a web of application servers and browsers to a web of web services and rich clients.
The following are some examples of this change in the way of developing web applications:

\begin{itemize}
\item The interest in Javascript execution engines by browser developers like Mozilla,
Apple or Google. 
\item The raise of rich Javascript frameworks based on the model view controller architecture like Sproutcore or
  Capuccino
\item The release of rich internet application development platforms like Adobe Flex, Java FX or Microsoft Silverligth.
\item The interest in key value stores as an alternative to relational data bases
\end{itemize}


Services must be carefully designed to expose data in an efficient way. Furthermore, today's web applications are built on top of data
from multiple providers, services like Google Maps or Twitter, are mashed up with other data to provide innovative
functionalities to the users. Besides, web agents are not anymore restricted to web browsers, a new generation of mobile
devices connected to the internet are consuming web data in mobile devices conforming what has been described as the
diffuse web \cite{conf/coordination/Serrano09}.\\

As important as the access to web data from web clients is the need of exchanging data from different providers. To
offer actual value to web users, interoperability for web data must be achieved, but the current situation shows data
providers using ad hoc data schemas and web services.\\

Semantic web technologies can be used to provide an easy and flexible solution to the problem of the generation, and
exchange of open data in the web. Semantic technologies can be also used in conjunction with new data persistence tools to
provide a powerful alternative to traditional relational data base driven web applications.\\

If easy ways of inserting web technologies in nowadays web development stacks are provided to web developers, it could
boost the adoption of web technologies and constitute the foundation for more advanced layers of semantic technology. 


\section{Goals of this work}

In this document we describe how pragmatic use of web technologies in current web applications can be
achieved through the use of what we have named RESTful semantic web services.\\

The plan of this document starts with the description of a very simple formal model that can be used to describe semantic web data
providers and consumers. Then, a specification of this model is built using standard semantic web technologies. This
specification is transformed later into an actual implementation of a client and server software libraries that can be
used to build semantic web applications with RESTful semantic web services. Finally, we describe one of
the applications that can be built with the described software libraries. It can serve as a sample of the kind of
applications that can be built with the technologies introduced.\\

Our main ambition is to propose feasible technological solutions for current day web development problems, for instance,
the open exchange of data between web services and rich internet clients. This solution uses semantic web technologies as a mean for solving these problems
but as a side effect, it conforms a basic layer where the whole semantic web stack of technologies could be delivered later. This technological
solution must be built on top of a valid theoretical base, useful for the description and design of web architectures, and must
adhere itself to the simplicity of the REST principles. Focusing on simplicity and productivity, without rejecting the use of standard semantic technologies, we try to provide
an attractive platform for developers that could help to boost the growth of the Semantic Web.